
\providecommand{\heading}[1]{\section{#1}}
\providecommand{\subheading}[1]{\subsection{#1}}

Education is the lifeblood of any profession. 
The field of cybersecurity is no exception to this. %%
Despite this necessity, there is little consensus among cybersecurity practitioners about what pedagogical methods are most effective.

Drawing from the broader field of Computer Science education, there is broad agreement regarding the utility and effectiveness of hands-on work. %%
However, this is where the agreement begins and ends. %%
There exists several competing schools of thought regarding how hands-on work should be applied and incorporated into educational settings in order to maximize learning outcomes. %%
Among these schools of thought, three stand out. %%
These schools of thought can be arranged along a spectrum. %%
One end of the spectrum suggesting that all instructor guidance should be minimized. %%
Conversely, the other end of the spectrum calls for instructor guidance to be maximized. %%
In between these two stances exists a compromise option, which seeks to emphasize the supposed benefits of both approaches.

This thesis seeks to qualitatively evaluate the merits of the three aforementioned approaches. %%
Two experiments were conducted, where three groups of students were given the same task, but each was provided a different set of instructions corresponding to the different design philosophies. %%
While there are a number of important caveats deriving from how the study was conducted, the results largely corroborate the conclusion that a maximal-guidance approach is the most practicable when used in the context of higher education. 

Educational literature qualitatively and quantitatively explores the merits of different approaches to designing hands-on activities. %%
Meanwhile, cybersecurity educators frequently develop and employ educational activities in their classrooms without significant consideration for the pedagogy underlying these activities. %%
This research unifies the two fields, demonstrating how such considerations can inform the development of cybersecurity education. 
