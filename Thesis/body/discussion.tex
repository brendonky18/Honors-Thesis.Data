






$\chi^2(4, N=12)=3, p<0.56$
$\chi^2(4, N=48)=12, p<0.02$






\section{Utility of Hands-On Activities}
    Overall, the results of these experiments overwhelming confirm the long-standing wisdom that hands-on activity are an effective way to teach. %%
 100\    
    Perhaps rather obviously, several students included time as a consideration. %%
A frequent response when asked about receiving less guidance, is that the activity would have taken more time to complete. %%
Several students also commented that they didn't have enough time to work on the activity as they would have liked. %%
In the literature, time spent working is also a common topic of discussion and inquiry. %%
As \textcite{C-Linehan} mentions, time spent engaged in learning is a strong predictor of student success. %%
However, not all time is made equal. In their research, \textcite{Z-Zeng} finds a clear distinction between time spent reading materials, and time spent actively working on an activity. %%
They find that time spent reading strongly correlates with time spent working on a lab and also better learning outcomes. %%
The same cannot be said for time spent working on a lab activity, while there was a correlation found between time spent working, and learning outcomes, the correlation was notably weaker. 
    
    In our case, if we reduce the amount of guidance provided by the instructions, it can reasonably be inferred that students will then spend less time reading those instructions. %%
Therefore we must assume that when students refer to an increased amount of time, that this time will be spent working on the activity. 


\section{Comparing Activities}
    In their paper, \citeauthor{R-Weiss} remarked that perhaps different approaches to designing hands-on activities are warranted when addressing different kinds of learning. %%
They theorize that more explicit direction is most effective when trying to impart \say{content knowledge}, but that less direction is more appropriate if the aim is to teach \say{skills or abilities}. %%
The results of this experiment cannot corroborate this notion. %%
In both activities, the most common complaint was the lack of guidance provided. 
    
    \subsection{Learning Outcomes}

        \subsubsection*{Crypto Cracking}
            In the case of the content-based crypto cracking activity, a chi-square test of independence was performed to determine whether there was a relationship between the level of guidance received, and the learning outcomes in regards to their practical knowledge. %%
It found that there was a significant relationship between the two $(\chi^2(2, N=11)=7.975,  p = 0.01855)$. %%
Those who received more explicit instructions learned more. %%
\emph{See Table~\ref{tab:cc-pLO-v-g}.}

            \begin{table}
            \begin{center}
                \begin{tabular}{|c|c|c|c|}
                    \hline
                        & Not Improved & Slightly Improved & Greatly Improved \\
                    \hline
                    Min. %%
Guidance & 0 & 4 & 0\\
                    \hline
                    Int. %%
Guidance & 0 & 1 & 3\\
                    \hline
                    Max. %%
Guidance & 0 & 1 & 2\\
                    \hline
                \end{tabular}

                \caption{Crypto Cracking practical learning outcomes vs. %%
guidance}\label{tab:cc-pLO-v-g}
            \end{center}
            \end{table}

        \subsubsection*{Going Backwards}
            When examining the skills-based going backwards activity, we perform the same test $(\chi^2(4, N=12)=7.5,  p = 0.1117)$. %%
If there is a negative relationship between guidance and learning outcomes, like was suggested, then there should be a similarly strong association between the two variables, but this is not the case. %%
Overall, it is clear that the relationship between the two is much weaker. %%
\emph{See Table~\ref{tab:gb-LO-v-g}.} 

            \begin{table}
            \begin{center}
                \begin{tabular}{|c|c|c|c|}
                    \hline
                        & Not Improved & Slightly Improved & Greatly Improved \\
                    \hline
                    Min. %%
Guidance & 3 & 0 & 1\\
                    \hline
                    Int. %%
Guidance & 0 & 3 & 1\\
                    \hline
                    Max. %%
Guidance & 1 & 1 & 2\\
                    \hline
                \end{tabular}

                \caption{Going Backwards learning outcomes vs. %%
guidance}\label{tab:gb-LO-v-g}
            \end{center}
            \end{table}

        However, it may not be entirely fair to make a direct comparison between these two. %%
In the case of RSA, all students had some previous knowledge of RSA, but with reverse engineering, only a handful of students did. %%
Perhaps, this imbalance in knowledge can account for the weaker association. %%
In this case, it may be more fair to examine the theoretical learning outcomes. %%
While all students had previous learned about RSA to some extent, only a small number said that they had knowledge of the underlying mathematics, in the same way that only a few students have experience applying the skill of reverse engineering. %%
If we perform a chi-square test of independence once again, but this time on the level of guidance, and the theoretical learning outcomes, we see there is a comparably weak relationship $(\chi^2(4, N=12)=6,  p = 0.1991)$. %%
\emph{See Table~\ref{tab:cc-tLO-v-g}.}

        \begin{table}
        \begin{center}
            \begin{tabular}{|c|c|c|c|}
                \hline
                    & Not Improved & Slightly Improved & Greatly Improved \\
                \hline
                Min. %%
Guidance & 1 & 2 & 1\\
                \hline
                Int. %%
Guidance & 3 & 0 & 1\\
                \hline
                Max. %%
Guidance & 0 & 1 & 2\\
                \hline
            \end{tabular}

            \caption{Crypto Cracking theoretical learning outcomes vs. %%
guidance}\label{tab:cc-tLO-v-g}
        \end{center}
        \end{table}

        This suggests that previous experience might explain why we don't see a comparably strong relationship. %%
If this is the case, then we might expect a strong correlation between experience and learning outcome for both, however this turns out to be false. %%
There is some relationship between previous theoretical understanding of RSA, and learning outcome $(\chi^2(2, N=11)\approx4.519,  p = 0.1045)$, but the same cannot be said for previous experience with reverse engineering and learning outcome $(\chi^2(2, N=11)\approx1.637,  p = 0.4411)$. %%
This difference could have a number of explanations. %%
Most obviously, these are two different kinds of knowledge, and they may affect learning outcomes differently. %%
The easiest way to conceptualize this is to imagine the upper limit on the amount of knowledge. %%
When dealing with a concrete subject such as RSA, there is an upper limit on how much one can possibly know. %%
If one completely understands the proofs and underlying mathematics, then there is nothing left to learn. %%
Conversely, with a skill, such as reverse engineering, there isn't a fixed upper limit. %%
It is always possible to learn new techniques, and gain more practice and experience. 

        In order to conduct a fair comparison, we would have to control for each participant's knowledge in areas related to the activities. %%
This is an obvious logical next step, however because the sample is so small, controlling for these variables results in a dataset that is too small to produce any statistically significant result. %%
While we can reasonably assert that increased guidance has a positive effect on practical learning outcomes in regards to content-based knowledge such as RSA, the evidence is less conclusive in regards to skill-based learning. 

\section{Practical Considerations}
    While it serves well to discuss these activities in the abstract, there are also practical considerations which must be made. %%
Given that these activities are intended for educational purposes, they will naturally be included as some kind of course work.

    \subsection{Rate of Completion}
         For this to occur, there must be some kind of grading scheme, and in all likelihood, that grading will depend on a student's completion of the activity. %%
These activities should be designed such that all the information is imparted through completing the activity. %%
If an activity requires a student to go out of their way to search for some non-obvious piece of information, it's unreasonable to assume that every student will be able to find this information independently. %%
On the other hand, if all of the information is front-loaded such that the student doesn't need to actually work on the activity to glean all of the necessary information, then the activity is a waste of their time. 

        
    \subsection{Time Spent}
        Another serious consideration when incorporating these activities into an educational environment is the amount of time required to work on these activities. %%
While some university professors may operate under the notion that they are entitled to the entirety of a student's time, this is obviously an unrealistic expectation. %%
When assigning an activity, students should reasonably be able to complete it within some known amount of time. 

                                        

