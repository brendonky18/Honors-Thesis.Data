\providecommand{\heading}[1]{\section{#1}}
\providecommand{\subheading}[1]{\subsection{#1}}


% A statement of the goal of the paper: why the study was undertaken, or why the paper was written. Do not repeat the abstract. 
    % Attempt to marry educational and cybersecurity research. 


% Sufficient background information to allow the reader to understand the context and significance of the question you are trying to address. 
    % Identified a gap in current literature


% Proper acknowledgement of the previous work on which you are building. Sufficient references such that a reader could, by going to the library, achieve a sophisticated understanding of the context and significance of the question.
    % 

% The introduction should be focused on the thesis question(s).  All cited work should be directly relevant to the goals of the thesis.  This is not a place to summarize everything you have ever read on a subject.
    % 

% Explain the scope of your work, what will and will not be included. 
    % 

% A verbal "road map" or verbal "table of contents" guiding the reader to what lies ahead. 
    % 

% Is it obvious where introductory material ("old stuff") ends and your contribution ("new stuff") begins? 
    % 

Hands-on learning has a clearly demonstrated and well accepted role to play in higher education. What's less clear is how these activities should be designed in order to fully harness their educational potential. Within the domain of cybersecurity education, there has been a significant amount of attention given to the design and development of hands-on activities. Unfortunately, these designs are largely independent of and serious consideration for the underlying pedagogy. Although there is a large interest in the development of interactive educational materials, this energy has primarily been invested in improving the technical rather than educational aspects. For example, the \emph{SEED} \cite{W-Du} environment was designed to provide a wide variety of exercises from a single environment; \textcite{N-Eliot} describe a framework for a safe sandbox environment; \emph{CyberAware} introduces cybersecurity activities for mobile devices \cite{F-Giannakas}. All of these proposals offer little pedagogical justifications for the design decisions being made. 

This research examines these exact considerations, and how the design of cybersecurity educational activities can be informed by existing educational research. The current literature is split between three different camps, in regards to how much guidance should be given to students. This research will investigate how different amounts of guidance can impact performance outcomes for students, in order to determine how future hands-on activities can be designed. 
    