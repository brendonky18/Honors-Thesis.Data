In total, 12 students from the course participated in the study by working on both activities, and responding to the surveys afterwards, resulting in 24 surveys. Of these, only three had no previous experience with CTFs. Two students responded stating that they have never previously used hands on activities to learn something, however there were problems with this specific questions that will be addressed in the \nameref{ch:discussion} chapter. Unfortunately, this means that the results of this question are mostly meaningless. When asked to reflect on their experience using hands-on activities to learn cybersecurity topics, all respondents remarked positively on the use of activities in at least some conditions. In other terms, no respondents unequivocally or unconditionally felt that hands-on activities had no use in educating cybersecurity. Of the 12 participants, usage of the instructions was distributed exactly evenly; four used the minimal guidance instructions, four used the intermediate guidance instructions, and four used the maximal guidance instructions. 

The remaining results will be broken down based on the two different activities, and the two kinds of learning which they correspond to, as identified by \textcite{R-Weiss}. 

The full data set is available in Appendix~\ref{app:data}.

% \section{Content-Based Learning}
%     As explained in the \nameref{ch:methods} chapter, the \emph{crypto cracking} activity was designed to focus on a student's content knowledge of the RSA public key cryptosystem. 
    
%     \subsection*{Previous Experience}
%         When asked about their previous experience learning and working with RSA, only 5 participants claimed to have any extensive theoretical or practical knowledge of RSA. Of these, 3 had implemented and otherwise worked with RSA encryption, the other 2 had a strong understanding of the math behind RSA. 
        
%         Among the 8 remaining students, one had never learned about RSA previously, 3 had at least a basic theoretical understanding of RSA, 2 had previously learned and subsequently forgotten about RSA, and 1 student did not respond to the question. 

%     \subsection*{Activity Difficulty}
%         Students spent anywhere from 1 to 10 hours working on this activity. When asked whether they thought that same time could have been used to learn RSA more effectively, 5 students thought that their time would have been better spent learning with a different method, 6 thought that this was the most effective way to spent that time learning about RSA, and 1 student did not respond. 

%         Students were also asked to rate the difficulty of the activity. They were told that a rating of 1 meant that the activity was trivial to solve, and a rating of 5 meant that the activity was impossible to solve. 6 students gave the activity a difficulty rating of 3, 5 gave it a rating of 4, and only 1 rated it 5. 
        
%         Interestingly, of the 12 participants, only half were able to make any amount of progress; 3 were able to complete the first task, and the other 3 were able to complete the entire activity. 
        
%     \subsection*{Learning Outcomes}
%         When asked about how much the activities improved their theoretical understanding of RSA, 4 students said that it did not change at all, 4 said that their understanding improved slightly, and the remaining 4 said that it improved greatly. When asked about how the activity improved their understanding of RSA's use and implementation, only 1 said that it did not change; 6 said that it improved slightly, and 5 said that it improved greatly. 

%     \subsection*{Activity Design}
%         When asked about the design of the instructions, 7 students  wanted more explicit guidance, 3 said that the amount of guidance provided was sufficient, 1 said that more guidance would have helped in terms of learning knowledge, but hindered in learning skills, and 1 student said that more guidance would not have impacted their learning, either positively or negatively. Conversely, 10 students remarked negatively about the prospect of less guidance, 1 student was unsure about how it would impacted their learning, and the remaining student said that the provided instructions did not provide any guidance as they were, meaning that any reduction would not have made a discernable impact.

%         Participants were then given the opportunity to freely discuss what they liked, disliked, and what they would change about the activity and instructions they were provided. 4 students said that they would make no changes, and 7 asked for more explicit guidance. The remaining student was able to find a work around, which bypassed the intended solution, and commented how this made the challenge fairly simple.

%         Another recurring complaint involved technical issues with how the activity was implemented. In total, 7 students encountered issues where they were unable to access the challenge, or were confused by the custom protocol that was implemented. 

% \section{Skills-Based Learning}
%     The \emph{going backwards} activity focuses on a students skills in the area of software reverse engineering. 

%     \subsection*{Previous Experience}
%         4 participants had previous experience using reverse engineering, in order to understand the functionality of a program, library or other code base. 2 other students were familiar with the notion or concept of reverse engineering, 5 students had no familiarity with reverse engineering, and 1 student did not respond to the question.         

%     \subsection*{Activity Difficulty}
%         On this activity, students spent anywhere from 45 minutes up to 10 hours. Notably, 2 students did not state how much time they spent. They were again asked whether they thought they could have learned reverse engineering better in that time with a different approach. 6 believed that their time could have been better spent, 5 thought the activity was an effective use of their time, and 1 did not respond to the question. 

%         6 students were able to at least locate the default password, and correctly recognize that they were an MD5 hash, however of these, only 1 was then able to reverse the encoding and hash, connect to the FTP server, and capture the flag. The other 6 students were not able to make any significant progress towards solving the activity. 

%     \subsection*{Learning Outcomes}
%         When asked about how the activity improved their understanding of reverse engineering, 4 said that their understanding did not improve, 4 said that it improved slightly, and 4 said that their understanding improved greatly. 

%     \subsection*{Activity Design}

% \section{Analysis}

\section{Exceptional Cases}
    There are two individual cases which are interesting as they are exceptions to the parameters of this experiment. These cases will be examined individually so that these exceptions don't have to be repeatedly noted in later analysis and discussion. 

    One student approached the crypto cracking activity with no previous knowledge of RSA. As a prerequisite, CS 561 requires some form of network security course work, RSA is an essential part of network security so these activities are based in the assumption that students have at least a basic familiarity with RSA and public key cryptography. However, this is not the case for student 05. Perhaps unsurprisingly, this student had tremendous difficulty while working on the activity. Although they had commented that the use of hands-on activities in the course CS 561 had been useful and informational, their responses to the survey clearly demonstrates that this particular activity was not helpful to them. This student received the maximal guidance instructions, but stated that the amount of guidance was inadequate, and that they were lost and unsure how to proceed. They spent 3 hours working on the activity, but was unable to complete any of the tasks. While their knowledge of the theoretical aspects of RSA was slightly improved, their understanding of it's usage and implementation did not change. 

    The other exception involves student 12. The design of the materials did not account for unintended solutions as the design focused on students working through each task in order to learn. This student was able to find an unintended solution to the activity, which allowed them to bypass most of the work. It should also be noted that they were already quite familiar with the RSA. Using the intermediate guidance instruction set, they used an online tool which automatically cracked the public keys for them, bypassing the need for them to discover and exploit each key's weakness. While their knowledge of the theory was not changed, their knowledge of RSA's implementation was greatly improved.

\begin{table}
    \begin{center}
        \begin{tabular}[]{|c|c|c|c|}
            \hline
            ID
            \tablefootnote{Participant ID number} & 
            
            Instruction Set
            \tablefootnote{Instruction Set used by the student} & 

            Previous Experience
            \tablefootnote{When have they used hands-on activities to learn} &
            
            Hands-on Helpful?
            \tablefootnote{Do they think hands-on activities are a helpful way to learn}
            \\

            \hline
            01 & 0 & Used personally & Yes \\
            \hline
            10 & 0 & Used in class & Yes \\
            \hline
            11 & 0 & Used in class & Yes \\
            \hline
            03 & 0 & Used in class & Yes \\
            \hline
            02 & 1 & Used in class & Yes \\
            \hline
            09 & 1 & Used in class & Yes \\
            \hline
            04 & 1 & Used in class & Yes \\
            \hline
            12 & 1 & Used personally & Yes \\
            \hline
            05 & 2 & Used in class & Yes \\
            \hline
            06 & 2 & Used in class & Yes \\
            \hline
            08 & 2 & Used in class & Yes \\
            \hline
            07 & 2 & Used in class & Yes \\
            \hline
        \end{tabular}
        \caption{Background Information}\label{tab:bg-info}
    \end{center}
\end{table}
\begin{table}
    \begin{center}
        \begin{tabular}[]{|c|c|c|c|c|c|}
            \hline
            ID
            \tablefootnote{Student ID} & 

            Progress
            \tablefootnote{How much of the activity was the student able to solve.} & 

            Time
            \tablefootnote{The amount of time spent workong on the activity by the student. \emph{If the student entered a range, the average value will be used and represented in (parentheses)}.} & 

            Bad use of time?
            \tablefootnote{Did the student think that the time they spent working on this activity was a less effective way of learning about RSA.} &
            
            Theo. LO.
            \tablefootnote{Amount of improvement in theoretical understanding of RSA} &

            Prac. LO.
            \tablefootnote{amount of improvement in practical understanding of RSA}
            \\
            \hline
            01 & None & 5 & Yes & Slight & Slight \\
            \hline
            10 & None & 2 & Yes & None & Slight \\
            \hline
            11 & None & 3 & | & Great & Slight \\
            \hline
            03 & User 1 pw & 2 & No & Slight & Slight \\
            \hline
            02 & None & 1 & Yes & None & Slight \\
            \hline
            09 & User 1 pw & 8-10 (9) & None & None & Great \\
            \hline
            04 & User 3 pw & 10 & Yes & Great & Great \\
            \hline
            12 & User 3 pw & 6 & No & None & Great \\
            \hline
            05 & None & 3 & Yes & Great & Great \\
            \hline
            06 & None & 6-8 (7) & No & Slight & None \\
            \hline
            08 & User 1 pw & 2 & No & Slight & Slight \\
            \hline
            07 & User 3 pw & 7 & No & Great & Great \\
            \hline
        \end{tabular}
        \caption{\emph{Crypto Cracking} Student Performance}\label{tab:cc-performance}
    \end{center}
\end{table}

\begin{table}
    \begin{center}
        \begin{tabular}[]{|c|c|c|c|c|c|c|c|}
            \hline
            ID & 

            Used
            \tablefootnote{Has the student ever used, implemented, or otherwise worked with RSA} &

            Theory
            \tablefootnote{Is the student familiar with the theory/math behind RSA, except for student 05, all students have had at least a basic introduction to the concepts of RSA and public key cryptography} &

            More G.
            \tablefootnote{Would more guidance improve their learning experience} &

            Less G.
            \tablefootnote{Would less guidance worsen their learning experience} &

            Liked Act.
            \tablefootnote{Did the student like the activity overall} &
            
            Tech. Iss.
            \tablefootnote{Did the student encounter technical issues} & 
            
            Time
            \tablefootnote{Was the amount of time an issue for the student?}
            \\
            \hline
            01 & Yes & No  & Yes & Yes & Yes & No  & No  \\
            \hline
            10 & Yes & Yes & Yes & Yes & Yes & Yes & No  \\
            \hline
            11 &  |  &  |  & Yes & Yes & No  & Yes & No  \\
            \hline
            03 & No  & Yes & Yes & Yes & Yes & No  & Yes  \\
            \hline
            02 & No  & No  & Yes & Yes & Yes & No  & No  \\
            \hline
            09 & Yes & Yes & Yes & Yes & Yes & Yes & No  \\
            \hline
            04 & No  & No  & \emph{Yes}\tablefootnote{This student said \say{More guidance would be complete hand-holding, and will help me learn but hinder my ability to figure out solutions on my own.}} & Yes & Yes &  Yes & No  \\
            \hline
            12 & Yes & Yes & \emph{No}\tablefootnote{This student said that increasing the guidance would have \say{No impact} on their learning}  & Yes & Yes &  Yes &  No  \\
            \hline
            05 & \emph{No}  & \emph{No}  & Yes & Yes & Yes & Yes & No  \\
            \hline
            06 & No  & No  & No  & Yes & Yes & Yes & No  \\
            \hline
            08 & No  & No  & No  & Yes & Yes & No  & No  \\
            \hline
            07 & No  & No  & No  & Yes & Yes & No  & No  \\
            \hline
        \end{tabular}
        \caption{\emph{Crypto Cracking} Student Feedback}\label{tab:cc-feedback}
    \end{center}
\end{table}

\begin{table}
    \begin{center}
        \begin{tabular}[]{|c|c|c|c|c|c|c|}
            \hline
            ID & 

            Progress & 

            Time & 

            Bad use of time?
            \tablefootnote{Did the student think that the time they spent working on this activity was a less effective way of learning about reverse engineering.} &
            
            Learning Outcome
            \tablefootnote{Amount of improvement in understanding of reverse engineering}
            \\
            \hline
            01 & Got password & 3 & No & None \\
            \hline
            10 & None & 0.75 & Yes & None \\
            \hline
            11 & Got Hash & | & Yes & None \\
            \hline
            03 & None & 2 & Yes & Great \\
            \hline
            02 & Got hash & 1 & Yes & Slight \\
            \hline
            09 & None & 3 & No & Slight \\
            \hline
            04 & Got flag & | & No & Slight \\
            \hline
            12 & Got password & 3 & No & \emph{Great}\tablefootnote{Student said \say{My understanding has definitely improved on how to read the relevant sections of code to figure out what is vulnerable.}} \\
            \hline
            05 & None & 3 & Yes & None \\
            \hline
            06 & Got flag & 8-10 (9) & Yes & Great \\
            \hline
            08 & None & 2-3 (2.5) & No & Slight \\
            \hline
            07 & Got flag & 9 & | & Great \\
            \hline
        \end{tabular}
        \caption{\emph{Going Backwards} Student Performance}\label{tab:gb-performance}
    \end{center}
\end{table}


\begin{table}
    \begin{center}
        \begin{tabular}[]{|c|c|c|c|c|c|c|}
            \hline
            ID & 

            Used
            \tablefootnote{Has the student ever used reverse engineering} &

            More G. &

            Less G. &

            Liked Act. &
            
            Tech. Iss. & 
            
            Time \\
            \hline
            01 & Yes & Yes & \emph{No}\tablefootnote{Instructions were already unhelpful, so less would not have been worse} & Yes & Yes & No  \\\hline
            10 & No  & Yes & Yes & Yes & No  & No  \\\hline
            11 & No  & Yes & Yes & Yes & No  & No  \\\hline
            03 & No  & Yes & Yes & Yes & No  & No  \\\hline
            02 & Yes & \emph{No}\tablefootnote{Student said \say{I may have completed it faster but I feel I was on a way to complete it.}} & Yes & Yes & Yes & Yes  \\\hline
            09 & No  & Yes & Yes & Yes & No  & No  \\\hline
            04 & No  & Yes & Yes & Yes & No  & No  \\\hline
            12 & Yes & No  & Yes & Yes & Yes & No  \\\hline
            05 & No  & Yes & Yes & Yes & No  & No  \\\hline
            06 & Yes & Yes & Yes & Yes & Yes & No  \\\hline
            08 & No  & \emph{No}\tablefootnote{Student said \say{No impact} either way}  & Yes & Yes & No  & No  \\\hline
            07 &  |  & No  & Yes & Yes & No  & No  \\\hline
        \end{tabular}
        \caption{\emph{Going Backwards} Student Feedback}\label{tab:gb-feedback}
    \end{center}
\end{table}